\begin{tabular}
\end{tabular}


%%Findability 
\begin{tabular}{|m{2.5cm}|m{3cm}|m{1cm}|m{1cm}|p{8cm}|}
\hline \textbf{Type}&\textbf{Metric}&\textbf{ID}&\textbf{Values}&\textbf{Description}\\ \hline
\textbf{Content support}&Persistent identifiers&FCS1&y/n&A system is assessed yes (y) if it automatically assigns a persistent identifier to a research artefact.\\ \hline
\textbf{}&Generates DOIs&FCS2&y/n&A system is assessed yes (y) if, in addition to a possible unique identifier local to the system, it also supports the generation of a DOI for a research artefact.\\ \hline
\textbf{}&External identifiers&FCS3&y/n&A system is assessed yes (y) if it allows the import of idenfiers of the research artefact, external to itself.\\ \hline
\textbf{}&Dereferenceable identifiers&FCS4&y/n&A system is assessed yes (y) if it provides a unique identifier for research artefacts that is dereferenceable to the artefact's location in the system.\\ \hline
\textbf{Content access}&Indexes and searches metadata&FCA1&y/n&A system is assessed yes (y) if it enables search over the metadata of the research artefact.\\ \hline
\textbf{}&Indexes and searches content&FCA2&y/n&A system is assessed yes (y) if it enables search over the content of the research artefact.\\ \hline
\textbf{}&Advanced search features&FCA3&y/n&A system is assessed yes (y) if it provides advanced search features and, if so, justify by listing the features.\\ \hline
\end{tabular}


%%Accessibility
\begin{tabular}{|m{2.5cm}|m{3cm}|m{1cm}|m{1cm}|p{8cm}|}
\hline \textbf{Type}&\textbf{Metric}&\textbf{ID}&\textbf{Values}&\textbf{Description}\\ \hline
\textbf{Content language}&Language support&ACL1&y/n&A system is assessed as providing language support (y) if its interface is available in more than one language.\\ \hline
\textbf{}&Protocols and APIs supported&ACL2&y/n&A system is assessed as providing protocols and APIs (y) if it makes available at least one protocol and/or API for access to research artefacts.\\ \hline
\textbf{Content availability}&Long term preservation&ACA1&y/n&A system is assesssed yes (y) if it offers a convenient mechanism to perform a long term preservation beyond a simple database backup.\\ \hline
\textbf{}&Availability&ACA2&h/m/l&A system is assed high if the uptime is $\geq$ 99,9\%, it is assessed medium if the uptime is between 99,9\% and 99\% and low if it is below 99\% .\\ \hline
\textbf{}&Access control&ACA3&open, closed, on-request&A system is assessed open, if it makes research artefacts public by default. It is assessed as closed if it is primary designed for closed repositories. It is assessed on-request if it offers an on-request feature.\\ \hline
\end{tabular}


%%Interoperability
\begin{tabular}{|m{2.5cm}|m{3cm}|m{1cm}|m{1cm}|p{8cm}|}
\hline \textbf{Type}&\textbf{Metric}&\textbf{ID}&\textbf{Values}&\textbf{Description}\\ \hline
\textbf{Content}&Standard format for metadata&IC1&y/n&A system is assessed yes (y) if it employs a standardised and established format for providing the metadata.\\ \hline
\textbf{}&Standard formats for content&IC2&y/n&A system is assessed yes (y) if it supports the provision of data in common and standard file formats.\\ \hline
\textbf{}&Persistent identifiers&IC3&y/n&A system is assessed yes (y) if it assigns a persistent identifier (e.g. a stable URL) to the research artefacts.\\ \hline
\textbf{}&Custom metadata&IC4&y/n&A system is assessed yes (y) if it allows to customize, extend and limit the metadata schema/format.\\ \hline
\textbf{}&Linking of metadata and content&IC5&y/n&A system is assessed yes (y) if it supports the creation and publication of semantic links between different metadata and/or research artefacts. \\ \hline
\textbf{Content interaction}&Import and upload of metadata and content&ICI1&y/p/n&A system is assessed yes (y) if it supports the import and upload of metadata and data via multiple means, e.g. a web frontend or a standard API. It is assessed partially (p) if it only supports the provision via a web interface.\\ \hline
\textbf{}&Export and download of metadata and content&ICI2&y/p/n&A system is assessed yes (y) if it supports the export and download of metadata and data via multiple means, e.g. a web frontend or a standard API. It is assessed partially (p) if it only supports the download via a web interface.\\ \hline
\textbf{}&Custom submission process&ICI3&y/n&A system is assessed yes (y) it it supports the creation and maintenance of a customised submission process, including fine-grain access control and role assignment.\\ \hline
\textbf{}&Data federation &ICI4&y/n&A system is assessed yes (y) if it provides a built-in mechanism to make the metadata and data in other repositories (running the same system) available. \\ \hline
\end{tabular}


%%Reusability
\begin{tabular}{|m{2.5cm}|m{3cm}|m{1cm}|m{1cm}|p{8cm}|}
\hline \textbf{Type}&\textbf{Metric}&\textbf{ID}&\textbf{Values}&\textbf{Description}\\ \hline
\textbf{Content depositing}&Licence support&RCD1&y/l/n&A system is assessed high (h) if it provides a highly usable and easy-to-understand mechanism to select a fitting licence for an artefact. It is assessed limited (l), if it at least provides a customizable controlled list of licences.\\ \hline
\textbf{Content access}&Licence support&RCA1&y/l/n&A system is assessed high (h) if it provides an easy-to-understand and human-readable description of the terms of use. It is assessed limited (l), if it at least provides a link to the applied licence.\\ \hline
\textbf{Content support}&Data&RCS1&y/n&A system is assessed as possessing the feature (y) if it allows to access the actual content in a structured and machine-readable manner, if the data format supports it, for example, if a user can access the content of a CSV via an API.\\ \hline
\end{tabular}


%%Engagement
\begin{tabular}{|m{2.5cm}|m{3cm}|m{1cm}|m{1cm}|p{8cm}|}
\hline \textbf{Type}&\textbf{Metric}&\textbf{ID}&\textbf{Values}&\textbf{Description}\\ \hline
\textbf{Usability and ease of use}&Support and documentation&EUA1&h/m/l&A system is assessed high (h) if the documentation and support are of high quality and reachability.\\ \hline
\textbf{}&Usability&EUA2&h/m/l&A system is assessed according to this metrics based on usability testing carried out.\\ \hline
\textbf{Engagement support}&Push notifications&EES1&y/n&A system is assessed as possessing the feature (y) if it includes mechanisms to notify users through mobile apps, email, or other mechanisms.\\ \hline
\textbf{}&Publication workflow support&EES2&y/n&A system is assessed as possessing the feature (y) if it provides a way to customise and manage the publication/deposit workflow.\\ \hline
\textbf{}&Visualisation tools&EES3&y/n&A system is assessed as possessing the feature (y) if it includes ways to visualise the content of publications or datasets, at least for some formats.\\ \hline
\textbf{}&Analysis tools&EES4&y/n&A system is assessed as possessing the feature (y) if it includes mechanisms to analyse the data in deposited research artefacts, including basic statistical analysis or more advanced methods. \\ \hline
\end{tabular}


%%Social connections
\begin{tabular}{|m{2.5cm}|m{3cm}|m{1cm}|m{1cm}|p{8cm}|}
\hline \textbf{Type}&\textbf{Metric}&\textbf{ID}&\textbf{Values}&\textbf{Description}\\ \hline
\textbf{Reflecting the social context}&Theming / branding&SRT1&y/n&A system is assessed as possessing the feature (y) if it enables the publisher or administrator to change the aspect of pages on the system to reflect institutional affiliation.\\ \hline
\textbf{}&Creation of collections&SRT2&y/n&A system is assessed as possessing the feature (y) if it includes ways for users to create, name, and publish arbitrary sets of research artefacts. \\ \hline
\textbf{}&Individual researcher profiles&SRT3&y/n&A system is assessed as possessing the feature (y) if it provides pages for individual researchers, including at least the research artefacts they have authored/published.\\ \hline
\textbf{Creating new social connections}&Like button&SCN1&y/n&A system is assessed as possessing the feature (y) if it gives the ability to users to provide simple positive feedback ("like") on research artefacts. \\ \hline
\textbf{}&Comments&SCN2&y/n&A system is assessed as possessing the feature (y) if it enables users to comment on individual research artefacts.\\ \hline
\textbf{}&Sharing&SCN3&y/n&A system is assessed as possessing the feature (y) if it provides ways to share research artefacts with other users, on the system or other platforms (e.g. social media). \\ \hline
\textbf{}&Following&SCN4&y/n&A system is assessed as possessing the feature (y) if it enables users to follow, and receive updates from, other users, research artefacts, collections, institutions, etc.\\ \hline
\textbf{}&Discussion forums&SCN5&y/n&A system is assessed as possessing the feature (y) if it includes discussion forums for users and/or for communication with/from the administrators.\\ \hline
\textbf{}&Development community&SCN6&h/m/l/c&For a proprietary, closed system, the assessement should be "closed" (c). For an open system, assessement is based on the frequency of activities in the development community (daily/weekly updates: high, monthly/quaterly updates: medium, less: low)\\ \hline
\end{tabular}


%%Trust
\begin{tabular}{|m{2.5cm}|m{3cm}|m{1cm}|m{1cm}|p{8cm}|}
\hline \textbf{Type}&\textbf{Metric}&\textbf{ID}&\textbf{Values}&\textbf{Description}\\ \hline
\textbf{Content}&Authentication&TC1&y/n&A system is assessed as possessing the feature (y) if it includes an authentication mechanism for users.\\ \hline
\textbf{}&Gate keeping&TC2&y/n&A system is assessed as possessing the feature (y) if it provides a review functionality during submission by data stewards or other permitted organizational users. \\ \hline
\textbf{}&Review feature&TC3&y/n&A system is assessed as possessing the feature (y) if it includes a functionality for writing reviews on specific research artefacts. \\ \hline
\textbf{Content support}&Indicator&TCS1&n/s/a&A system is assessed as possessing the feature (s) if it records usage statistics and (a) if it provides advanced research indicators like h-Index or AltMetrics.\\ \hline
\textbf{}&Long term preservation&TCS2&date&Date at which the first available artefact was deposited on the platform. \\ \hline
\textbf{System}&Open Source software and libraries&TS1&y/n&A system is assessed as possessing the feature (y) if the underlying system is open source. \\ \hline
\textbf{}&Uptake&TS2&h/m/l&A system is assessed as having high uptake (h) if it is used by a thousands of active users each month, medium uptake (m) with hundreds of active users, and low uptake (l) with lower numbers of active users.\\ \hline
\textbf{GDPR}&System backup&TG1&y/n&A system is assessed as possessing the feature (y) if it includes automated backups in a constant time interval.\\ \hline
\textbf{}&Right of information&TG2&y/n&A system is assessed as possessing the feature (y) if it provides a function to get all data about one user. \\ \hline
\textbf{}&Data deletion&TG3&y/n&A system is assessed as possessing the feature (y) if it provides a function to delete all data about one user. \\ \hline
\textbf{}&Agreement per data management process&TG4&y/n&A system is assessed as possessing the feature (y) if it allows the user to opt-in, agreeing to single personal data management processes individually.\\ \hline
\textbf{}&Portable and secure data exchange format&TG5&y/n&A system is assessed as possessing the feature (y) if it provides all personal data in an open standard format (e.g. HTML, TXT, PDF).\\ \hline
\textbf{}&Protection against data leaks&TG6&y/n&A system is assessed as possessing the feature (y) if it includes security tests for personal data. \\ \hline
\end{tabular}
